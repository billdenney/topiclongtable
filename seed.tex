\documentclass[11pt]{article}
\usepackage{lipsum} % for dummy text only
\usepackage{longtable}
\setlength{\arrayrulewidth}{1.2pt}

\usepackage[
  paperwidth=3in,
  paperheight=3in,
  margin=.5in,
  showframe
]{geometry}

\usepackage{zref-abspage}[2010/03/26]


\usepackage{array}% http://ctan.org/pkg/array


\usepackage{xparse}

\newcounter{toppdepth}

%% \DeclareDocumentCommand{\Topic}{mo}{
%%   \IfNoValueTF{#2}{
%%     \ifcsname toppL#1\endcsname
%%       U%% \expandafter\def\csname toppL#1\endcsname{#2}%
%%     \else
%%       D%% \expandafter\def\csname toppL#1\endcsname{L#1}%
%%     \fi
%%   }{
%%     \expandafter\let\csname toppL1\endcsname\undefined%
%%     \expandafter\let\csname toppL2\endcsname\undefined%
%%     % RESET STUFF
%%     \expandafter\def\csname toppL#1\endcsname{#2}%
%%   }

%%   \csname toppL#1\endcsname
%% }


\newcounter{tabcol}\newcounter{tabrow}
\newcolumntype{C}{>{\stepcounter{tabcol}}}



\usepackage{expl3}

\ExplSyntaxOn

\seq_new:N \TpcLbl
\seq_new:N \TpcIdx
\seq_new:N \TpcNew

\cs_new:Npn \SetTopic #1#2 {
  \seq_if_in:NnTF \TpcIdx {#1} {
    \bool_do_until:Nn \l_tmpa_bool {
      \seq_gpop:NN \TpcLbl \l_tmpa_tl
      \seq_gpop:NN \TpcIdx \l_tmpa_tl
      \tl_set:Nn \l_tmpb_tl {#1}
      \bool_gset:Nn \l_tmpa_bool { \tl_if_eq_p:NN \l_tmpa_tl \l_tmpb_tl }
    }
  } {}
  \seq_gpush:Nn \TpcIdx {#1}
  \seq_gpush:Nn \TpcLbl {#2}
  \seq_get:NN \TpcLbl \CurTpcLbl
  \CurTpcLbl
}

\msg_new:nnnn {tabloid} {invalid-index} {
  You~tried~to~use~an~uninitialized~topic~index~\msg_line_context:.
} {TODO}

\cs_new:Npn \GetTopic #1 {
  \seq_if_in:NnTF \TpcIdx {#1} {
    \seq_set_eq:NN \l_tmpa_seq \TpcIdx
    \seq_set_eq:NN \l_tmpb_seq \TpcLbl

    \bool_do_until:Nn \l_tmpa_bool {
      \seq_gpop:NN \l_tmpa_seq \l_tmpa_tl
      \tl_set:Nn \l_tmpb_tl {#1}
      \bool_gset:Nn \l_tmpa_bool { \tl_if_eq_p:NN \l_tmpa_tl \l_tmpb_tl }

      \seq_gpop:NN \l_tmpb_seq \l_tmpa_tl
    }
  } {
    \msg_error:nn {tabloid} {invalid-index}
  }
  %% \l_tmpa_tl
}

\NewDocumentCommand{\Topic}{mo}{%
  \IfValueTF{#2}{\SetTopic{#1}{#2}}{\GetTopic{#1}}%
}


%% \ExplSyntaxOff

\begin{document}

%% \begin{tabular}{>{\stepcounter{tabrow}\setcounter{tabcol}{1}}CCC}
%% \end{tabular}

\begin{longtable}{|l|l|l|l|}                              \cline{1-4}
  \Topic{1}[T1] & \Topic{2}[ST1] & \Topic{3}[SST1] & A \\ \cline{3-4}
  \Topic{1}     & \Topic{2}      & \Topic{3}[SST2] & B \\ \cline{2-4}
  \Topic{1}     & \Topic{2}[ST2] & \Topic{3}[SST3] & C \\ \cline{1-4}
  \Topic{1}[T2] & \Topic{2}[ST3] & \Topic{3}[SST4] & D \\ \cline{2-4}
  \Topic{1}     & \Topic{2}[ST4] & \Topic{3}[SST5] & E \\ \cline{4-4}
  \Topic{1}     & \Topic{2}      & \Topic{3}       & F \\ \cline{1-4}
\end{longtable}

%% \begin{longtable}{|p{3.2cm}|p{2.3cm}|p{10.5cm}|}
%%   \hline
%% \endfoot
%%   \hline
%%   \bfseries Head 1 & \bfseries Head 2 & \bfseries Head 3\\[5ex]
%%   \hline
%% \endhead
%%   \topic{Topic 1} & Subtopic A & There is some text here but could be small \\ %\cline{2-3}
%%   \lasttopic      & Subtopic B & \lipsum[1-3] \\ %\cline{2-3}
%%   \lasttopic      & Subtopic C & \lipsum[1-2] \\ %\cline{2-3}
%%   %\hline
%%   \topic{Topic 2} & Subtopic A & \lipsum[1-3] \\ %\cline{2-3}
%%   \lasttopic      & Subtopic B & \lipsum[1-2] \\ %\cline{2-3}
%%   \lasttopic      & Subtopic C & \lipsum[1-2] \\ %\cline{2-3}
%%   %\hline
%% \end{longtable}
\end{document}
