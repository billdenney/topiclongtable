\documentclass[full,kernel]{l3doc}
\begin{document}

\title{%
  The \pkg{topiclongtable} package%
  \thanks{Development of this package was sponsored by Human Predictions,
    LLC (\href{http://www.humanpredictions.com}{www.humanpredictions.com}).}\\
  Autocollapsing cells in longtables
}

\author{
  Paolo Brasolin\\
  \href{mailto:paolo.brasolin@gmail.com}{paolo.brasolin@gmail.com}
}

\date{2018/18/01 v1.2.0}

\maketitle

\begin{documentation}

Comma lists contain ordered data where items can be added to the left
or right end of the list.  This data type allows basic list
manipulations such as adding/removing items, applying a function to
every item, removing duplicate items, extracting a given item, using
the comma list with specified separators, and so on.  Sequences
(defined in \pkg{l3seq}) are safer, faster, and provide more features,
so they should often be preferred to comma lists.  Comma lists are
mostly useful when interfacing with \LaTeXe{} or other code that
expects or provides comma list data.

Several items can be added at once.  To ease input of comma lists from
data provided by a user outside an \cs{ExplSyntaxOn} \ldots{}
\cs{ExplSyntaxOff} block, spaces are removed from both sides of each
comma-delimited argument upon input.  Blank arguments are ignored, to
allow for trailing commas or repeated commas (which may otherwise
arise when concatenating comma lists \enquote{by hand}).  In addition,
a set of braces is removed if the result of space-trimming is braced:
this allows the storage of any item in a comma list.  For instance,
\begin{verbatim}
  \clist_new:N \l_my_clist
  \clist_put_left:Nn \l_my_clist { ~a~ , ~{b}~ , c~\d }
  \clist_put_right:Nn \l_my_clist { ~{e~} , , {{f}} , }
\end{verbatim}
results in |\l_my_clist| containing |a,b,c~\d,{e~},{{f}}| namely the
five items |a|, |b|, |c~\d|, |e~| and~|{f}|.  Comma lists normally do
not contain empty items so the following gives an empty comma list:
\begin{verbatim}
  \clist_clear_new:N \l_my_clist
  \clist_put_right:Nn \l_my_clist { , ~ , , }
  \clist_if_empty:NTF \l_my_clist { true } { false }
\end{verbatim}
and it leaves \texttt{true} in the input stream.  To include an
\enquote{unsafe} item (empty, or one that contains a comma, or starts
or ends with a space, or is a single brace group), surround it with
braces.

Almost all operations on comma lists are
noticeably slower than those on sequences so converting the data to
sequences using \cs{seq_set_from_clist:Nn} (see \pkg{l3seq}) may be
advisable if speed is important.  The exception is that
\cs{clist_if_in:NnTF} and \cs{clist_remove_duplicates:N} may be faster
than their sequence analogues for large lists.  However, these
functions work slowly for \enquote{unsafe} items that must be braced,
and may produce errors when their argument contains |{|, |}| or |#|
(assuming the usual \TeX{} category codes apply).  In addition, comma
lists cannot store quarks \cs{q_mark} or \cs{q_stop}.  The sequence
data type should thus certainly be preferred to comma lists to store
such items.

\section{Creating and initialising comma lists}

\begin{function}{\clist_new:N, \clist_new:c}
  \begin{syntax}
    \cs{clist_new:N} \meta{comma list}
  \end{syntax}
  Creates a new \meta{comma list} or raises an error if the name is
  already taken. The declaration is global. The \meta{comma list}
  initially contains no items.
\end{function}

\begin{function}[added = 2014-07-05]
  {
    \clist_const:Nn, \clist_const:Nx,
    \clist_const:cn, \clist_const:cx
  }
  \begin{syntax}
    \cs{clist_const:Nn} \meta{clist~var} \Arg{comma list}
  \end{syntax}
  Creates a new constant \meta{clist~var} or raises an error
  if the name is already taken. The value of the
  \meta{clist~var} is set globally to the
  \meta{comma list}.
\end{function}

\begin{function}
  {\clist_clear:N, \clist_clear:c, \clist_gclear:N, \clist_gclear:c}
  \begin{syntax}
    \cs{clist_clear:N} \meta{comma list}
  \end{syntax}
  Clears all items from the \meta{comma list}.
\end{function}

\begin{function}
  {
    \clist_clear_new:N,  \clist_clear_new:c,
    \clist_gclear_new:N, \clist_gclear_new:c
  }
  \begin{syntax}
    \cs{clist_clear_new:N} \meta{comma list}
  \end{syntax}
  Ensures that the \meta{comma list} exists globally by applying
  \cs{clist_new:N} if necessary, then applies
  \cs[index=clist_clear:N]{clist_(g)clear:N} to leave
  the list empty.
\end{function}

\begin{function}
  {
    \clist_set_eq:NN,  \clist_set_eq:cN,
    \clist_set_eq:Nc,  \clist_set_eq:cc,
    \clist_gset_eq:NN, \clist_gset_eq:cN,
    \clist_gset_eq:Nc, \clist_gset_eq:cc
  }
  \begin{syntax}
    \cs{clist_set_eq:NN} \meta{comma list_1} \meta{comma list_2}
  \end{syntax}
  Sets the content of \meta{comma list_1} equal to that of
  \meta{comma list_2}.
\end{function}

\begin{function}[added = 2014-07-17]
  {
    \clist_set_from_seq:NN,  \clist_set_from_seq:cN,
    \clist_set_from_seq:Nc,  \clist_set_from_seq:cc,
    \clist_gset_from_seq:NN, \clist_gset_from_seq:cN,
    \clist_gset_from_seq:Nc, \clist_gset_from_seq:cc
  }
  \begin{syntax}
    \cs{clist_set_from_seq:NN} \meta{comma list} \meta{sequence}
  \end{syntax}
  Converts the data in the \meta{sequence} into a \meta{comma list}:
  the original \meta{sequence} is unchanged.
  Items which contain either spaces or commas are surrounded by braces.
\end{function}

\begin{function}
  {
    \clist_concat:NNN,  \clist_concat:ccc,
    \clist_gconcat:NNN, \clist_gconcat:ccc
  }
  \begin{syntax}
    \cs{clist_concat:NNN} \meta{comma list_1} \meta{comma list_2} \meta{comma list_3}
  \end{syntax}
  Concatenates the content of \meta{comma list_2} and \meta{comma list_3}
  together and saves the result in \meta{comma list_1}. The items in
  \meta{comma list_2} are placed at the left side of the new comma list.
\end{function}

\begin{function}[EXP, pTF, added=2012-03-03]
  {\clist_if_exist:N, \clist_if_exist:c}
  \begin{syntax}
    \cs{clist_if_exist_p:N} \meta{comma list}
    \cs{clist_if_exist:NTF} \meta{comma list} \Arg{true code} \Arg{false code}
  \end{syntax}
  Tests whether the \meta{comma list} is currently defined.  This does
  not check that the \meta{comma list} really is a comma list.
\end{function}

\section{Adding data to comma lists}

\begin{function}[added = 2011-09-06]
  {
    \clist_set:Nn,  \clist_set:NV,
    \clist_set:No,  \clist_set:Nx,
    \clist_set:cn,  \clist_set:cV,
    \clist_set:co,  \clist_set:cx,
    \clist_gset:Nn, \clist_gset:NV,
    \clist_gset:No, \clist_gset:Nx,
    \clist_gset:cn, \clist_gset:cV,
    \clist_gset:co, \clist_gset:cx
  }
  \begin{syntax}
    \cs{clist_set:Nn} \meta{comma list} |{|\meta{item_1},\ldots{},\meta{item_n}|}|
  \end{syntax}
  Sets \meta{comma list} to contain the \meta{items},
  removing any previous content from the variable.
  Blank items are omitted, spaces are removed from both sides of each
  item, then a set of braces is removed if the resulting space-trimmed
  item is braced.
  To store some \meta{tokens} as a single \meta{item} even if the
  \meta{tokens} contain commas or spaces, add a set of braces:
  \cs{clist_set:Nn} \meta{comma list} |{| \Arg{tokens} |}|.
\end{function}

\begin{function}[updated = 2011-09-05]
  {
    \clist_put_left:Nn,  \clist_put_left:NV,
    \clist_put_left:No,  \clist_put_left:Nx,
    \clist_put_left:cn,  \clist_put_left:cV,
    \clist_put_left:co,  \clist_put_left:cx,
    \clist_gput_left:Nn, \clist_gput_left:NV,
    \clist_gput_left:No, \clist_gput_left:Nx,
    \clist_gput_left:cn, \clist_gput_left:cV,
    \clist_gput_left:co, \clist_gput_left:cx
  }
  \begin{syntax}
    \cs{clist_put_left:Nn} \meta{comma list} |{|\meta{item_1},\ldots{},\meta{item_n}|}|
  \end{syntax}
  Appends the \meta{items} to the left of the \meta{comma list}.
  Blank items are omitted, spaces are removed from both sides of each
  item, then a set of braces is removed if the resulting space-trimmed
  item is braced.
  To append some \meta{tokens} as a single \meta{item} even if the
  \meta{tokens} contain commas or spaces, add a set of braces:
  \cs{clist_put_left:Nn} \meta{comma list} |{| \Arg{tokens} |}|.
\end{function}

\begin{function}[updated = 2011-09-05]
  {
    \clist_put_right:Nn,  \clist_put_right:NV,
    \clist_put_right:No,  \clist_put_right:Nx,
    \clist_put_right:cn,  \clist_put_right:cV,
    \clist_put_right:co,  \clist_put_right:cx,
    \clist_gput_right:Nn, \clist_gput_right:NV,
    \clist_gput_right:No, \clist_gput_right:Nx,
    \clist_gput_right:cn, \clist_gput_right:cV,
    \clist_gput_right:co, \clist_gput_right:cx
  }
  \begin{syntax}
    \cs{clist_put_right:Nn} \meta{comma list} |{|\meta{item_1},\ldots{},\meta{item_n}|}|
  \end{syntax}
  Appends the \meta{items} to the right of the \meta{comma list}.
  Blank items are omitted, spaces are removed from both sides of each
  item, then a set of braces is removed if the resulting space-trimmed
  item is braced.
  To append some \meta{tokens} as a single \meta{item} even if the
  \meta{tokens} contain commas or spaces, add a set of braces:
  \cs{clist_put_right:Nn} \meta{comma list} |{| \Arg{tokens} |}|.
\end{function}

\section{Modifying comma lists}

While comma lists are normally used as ordered lists, it may be
necessary to modify the content. The functions here may be used
to update comma lists, while retaining the order of the unaffected
entries.

\begin{function}
  {
    \clist_remove_duplicates:N,  \clist_remove_duplicates:c,
    \clist_gremove_duplicates:N, \clist_gremove_duplicates:c
  }
  \begin{syntax}
    \cs{clist_remove_duplicates:N} \meta{comma list}
  \end{syntax}
  Removes duplicate items from the \meta{comma list}, leaving the
  left most copy of each item in the \meta{comma list}.  The \meta{item}
  comparison takes place on a token basis, as for \cs{tl_if_eq:nn(TF)}.
  \begin{texnote}
    This function iterates through every item in the \meta{comma list} and
    does a comparison with the \meta{items} already checked. It is therefore
    relatively slow with large comma lists.
    Furthermore, it may fail if any of the items in the
    \meta{comma list} contains |{|, |}|, or |#|
    (assuming the usual \TeX{} category codes apply).
  \end{texnote}
\end{function}

\begin{function}[updated = 2011-09-06]
  {
    \clist_remove_all:Nn,  \clist_remove_all:cn,
    \clist_gremove_all:Nn, \clist_gremove_all:cn
  }
  \begin{syntax}
    \cs{clist_remove_all:Nn} \meta{comma list} \Arg{item}
  \end{syntax}
  Removes every occurrence of \meta{item} from the \meta{comma list}.
  The \meta{item} comparison takes place on a token basis, as for
  \cs{tl_if_eq:nn(TF)}.
  \begin{texnote}
    The function may fail if the \meta{item} contains |{|, |}|, or |#|
    (assuming the usual \TeX{} category codes apply).
  \end{texnote}
\end{function}

\begin{function}[added = 2014-07-18]
  {
    \clist_reverse:N, \clist_reverse:c,
    \clist_greverse:N, \clist_greverse:c
  }
  \begin{syntax}
    \cs{clist_reverse:N} \meta{comma list}
  \end{syntax}
  Reverses the order of items stored in the \meta{comma list}.
\end{function}

\begin{function}[added = 2014-07-18]{\clist_reverse:n}
  \begin{syntax}
    \cs{clist_reverse:n} \Arg{comma list}
  \end{syntax}
  Leaves the items in the \meta{comma list} in the input stream in
  reverse order.  Contrarily to other what is done for other
  \texttt{n}-type \meta{comma list} arguments, braces and spaces are
  preserved by this process.
  \begin{texnote}
    The result is returned within \tn{unexpanded}, which means that the
    comma list does not expand further when appearing in an
    \texttt{x}-type argument expansion.
  \end{texnote}
\end{function}

\begin{function}[added = 2017-02-06]
  {\clist_sort:Nn, \clist_sort:cn, \clist_gsort:Nn, \clist_gsort:cn}
  \begin{syntax}
    \cs{clist_sort:Nn} \meta{clist var} \Arg{comparison code}
  \end{syntax}
  Sorts the items in the \meta{clist var} according to the
  \meta{comparison code}, and assigns the result to
  \meta{clist var}. The details of sorting comparison are
  described in Section~\ref{sec:l3sort:mech}.
\end{function}

\section{Comma list conditionals}

\begin{function}[EXP,pTF]{\clist_if_empty:N, \clist_if_empty:c}
  \begin{syntax}
    \cs{clist_if_empty_p:N} \meta{comma list}
    \cs{clist_if_empty:NTF} \meta{comma list} \Arg{true code} \Arg{false code}
  \end{syntax}
  Tests if the \meta{comma list} is empty (containing no items).
\end{function}

\begin{function}[EXP, pTF, added = 2014-07-05]{\clist_if_empty:n}
  \begin{syntax}
    \cs{clist_if_empty_p:n} \Arg{comma list}
    \cs{clist_if_empty:nTF} \Arg{comma list} \Arg{true code} \Arg{false code}
  \end{syntax}
  Tests if the \meta{comma list} is empty (containing no items).
  The rules for space trimming are as for other \texttt{n}-type
  comma-list functions, hence the comma list |{~,~,,~}| (without
  outer braces) is empty, while |{~,{},}| (without outer braces)
  contains one element, which happens to be empty: the comma-list
  is not empty.
\end{function}

\begin{function}[updated = 2011-09-06, TF]
  {
     \clist_if_in:Nn, \clist_if_in:NV, \clist_if_in:No,
     \clist_if_in:cn, \clist_if_in:cV, \clist_if_in:co,
     \clist_if_in:nn, \clist_if_in:nV, \clist_if_in:no
  }
  \begin{syntax}
    \cs{clist_if_in:NnTF} \meta{comma list} \Arg{item} \Arg{true code} \Arg{false code}
  \end{syntax}
  Tests if the \meta{item} is present in the \meta{comma list}.
  In the case of an \texttt{n}-type \meta{comma list}, the usual rules
  of space trimming and brace stripping apply.  Hence,
  \begin{verbatim}
    \clist_if_in:nnTF { a , {b}~ , {b} , c } { b } {true} {false}
  \end{verbatim}
  yields \texttt{true}.
  \begin{texnote}
    The function may fail if the \meta{item} contains |{|, |}|, or |#|
    (assuming the usual \TeX{} category codes apply).
  \end{texnote}
\end{function}

\section{Mapping to comma lists}

The functions described in this section apply a specified function
to each item of a comma list.

When the comma list is given explicitly, as an \texttt{n}-type argument,
spaces are trimmed around each item.
If the result of trimming spaces is empty, the item is ignored.
Otherwise, if the item is surrounded by braces, one set is removed,
and the result is passed to the mapped function. Thus, if the
comma list that is being mapped is \verb*|{a , {{b} }, ,{}, {c},}|
then the arguments passed to the mapped function are
`\verb*|a|', `\verb*|{b} |', an empty argument, and `\verb*|c|'.

When the comma list is given as an \texttt{N}-type argument, spaces
have already been trimmed on input, and items are simply stripped
of one set of braces if any. This case is more efficient than using
\texttt{n}-type comma lists.

\begin{function}[rEXP, updated = 2012-06-29]
  {\clist_map_function:NN, \clist_map_function:cN, \clist_map_function:nN}
  \begin{syntax}
    \cs{clist_map_function:NN} \meta{comma list} \meta{function}
  \end{syntax}
  Applies \meta{function} to every \meta{item} stored in the
  \meta{comma list}. The \meta{function} receives one argument for
  each iteration. The \meta{items} are returned from left to right.
  The function \cs{clist_map_inline:Nn} is in general more efficient
  than \cs{clist_map_function:NN}.
\end{function}

\begin{function}[updated = 2012-06-29]
  {\clist_map_inline:Nn, \clist_map_inline:cn, \clist_map_inline:nn}
  \begin{syntax}
    \cs{clist_map_inline:Nn} \meta{comma list} \Arg{inline function}
  \end{syntax}
  Applies \meta{inline function} to every \meta{item} stored
  within the \meta{comma list}. The \meta{inline function} should
  consist of code which receives the \meta{item} as |#1|.
  The \meta{items} are returned from left to right.
\end{function}

\begin{function}[updated = 2012-06-29]
  {\clist_map_variable:NNn, \clist_map_variable:cNn, \clist_map_variable:nNn}
  \begin{syntax}
    \cs{clist_map_variable:NNn} \meta{comma list} \meta{variable} \Arg{code}
  \end{syntax}
  Stores each \meta{item} of the \meta{comma list} in turn in the
  (token list) \meta{variable} and applies the \meta{code}.  The
  \meta{code} will usually make use of the \meta{variable}, but this
  is not enforced.  The assignments to the \meta{variable} are local.
  The \meta{items} are returned from left to right.
\end{function}

\begin{function}[rEXP, updated = 2012-06-29]{\clist_map_break:}
  \begin{syntax}
    \cs{clist_map_break:}
  \end{syntax}
  Used to terminate a \cs[no-index]{clist_map_\ldots{}} function before all
  entries in the \meta{comma list} have been processed. This
  normally takes place within a conditional statement, for example
  \begin{verbatim}
    \clist_map_inline:Nn \l_my_clist
      {
        \str_if_eq:nnTF { #1 } { bingo }
          { \clist_map_break: }
          {
            % Do something useful
          }
      }
  \end{verbatim}
  Use outside of a \cs[no-index]{clist_map_\ldots{}} scenario leads to low
  level \TeX{} errors.
  \begin{texnote}
    When the mapping is broken, additional tokens may be inserted
    before further items are taken
    from the input stream. This depends on the design of the mapping
    function.
  \end{texnote}
\end{function}

\begin{function}[updated = 2012-06-29, rEXP]{\clist_map_break:n}
  \begin{syntax}
    \cs{clist_map_break:n} \Arg{code}
  \end{syntax}
  Used to terminate a \cs[no-index]{clist_map_\ldots{}} function before all
  entries in the \meta{comma list} have been processed, inserting
  the \meta{code} after the mapping has ended. This
  normally takes place within a conditional statement, for example
  \begin{verbatim}
    \clist_map_inline:Nn \l_my_clist
      {
        \str_if_eq:nnTF { #1 } { bingo }
          { \clist_map_break:n { <code> } }
          {
            % Do something useful
          }
      }
  \end{verbatim}
  Use outside of a \cs[no-index]{clist_map_\ldots{}} scenario leads to low
  level \TeX{} errors.
  \begin{texnote}
    When the mapping is broken, additional tokens may be inserted
    before the \meta{code} is
    inserted into the input stream.
    This depends on the design of the mapping function.
  \end{texnote}
\end{function}

\begin{function}[EXP, added = 2012-07-13]
  {\clist_count:N, \clist_count:c, \clist_count:n}
  \begin{syntax}
    \cs{clist_count:N} \meta{comma list}
  \end{syntax}
  Leaves the number of items in the \meta{comma list} in the input
  stream as an \meta{integer denotation}. The total number of items
  in a \meta{comma list} includes those which are duplicates,
  \emph{i.e.}~every item in a \meta{comma list} is counted.
\end{function}

\section{Using the content of comma lists directly}

\begin{function}[EXP, added = 2013-05-26]{\clist_use:Nnnn, \clist_use:cnnn}
  \begin{syntax}
    \cs{clist_use:Nnnn} \meta{clist~var} \Arg{separator~between~two} \Arg{separator~between~more~than~two} \Arg{separator~between~final~two}
  \end{syntax}
  Places the contents of the \meta{clist~var} in the input stream,
  with the appropriate \meta{separator} between the items.  Namely, if
  the comma list has more than two items, the \meta{separator between
    more than two} is placed between each pair of items except the
  last, for which the \meta{separator between final two} is used.  If
  the comma list has exactly two items, then they are placed in the input
  stream separated by the \meta{separator between two}.  If the comma
  list has a single item, it is placed in the input stream, and a comma
  list with no items produces no output.  An error is raised if
  the variable does not exist or if it is invalid.

  For example,
  \begin{verbatim}
    \clist_set:Nn \l_tmpa_clist { a , b , , c , {de} , f }
    \clist_use:Nnnn \l_tmpa_clist { ~and~ } { ,~ } { ,~and~ }
  \end{verbatim}
  inserts \enquote{\texttt{a, b, c, de, and f}} in the input
  stream.  The first separator argument is not used in this case
  because the comma list has more than $2$ items.
  \begin{texnote}
    The result is returned within the \tn{unexpanded}
    primitive (\cs{exp_not:n}), which means that the \meta{items}
    do not expand further when appearing in an \texttt{x}-type
    argument expansion.
  \end{texnote}
\end{function}

\begin{function}[EXP, added = 2013-05-26]{\clist_use:Nn, \clist_use:cn}
  \begin{syntax}
    \cs{clist_use:Nn} \meta{clist~var} \Arg{separator}
  \end{syntax}
  Places the contents of the \meta{clist~var} in the input stream,
  with the \meta{separator} between the items. If the comma
  list has a single item, it is placed in the input stream, and a comma
  list with no items produces no output.  An error is raised if
  the variable does not exist or if it is invalid.

  For example,
  \begin{verbatim}
    \clist_set:Nn \l_tmpa_clist { a , b , , c , {de} , f }
    \clist_use:Nn \l_tmpa_clist { ~and~ }
  \end{verbatim}
  inserts \enquote{\texttt{a and b and c and de and f}} in the input
  stream.
  \begin{texnote}
    The result is returned within the \tn{unexpanded}
    primitive (\cs{exp_not:n}), which means that the \meta{items}
    do not expand further when appearing in an \texttt{x}-type
    argument expansion.
  \end{texnote}
\end{function}

\section{Comma lists as stacks}

Comma lists can be used as stacks, where data is pushed to and popped
from the top of the comma list. (The left of a comma list is the top, for
performance reasons.) The stack functions for comma lists are not
intended to be mixed with the general ordered data functions detailed
in the previous section: a comma list should either be used as an
ordered data type or as a stack, but not in both ways.

\begin{function}[updated = 2012-05-14]{\clist_get:NN, \clist_get:cN}
  \begin{syntax}
    \cs{clist_get:NN} \meta{comma list} \meta{token list variable}
  \end{syntax}
  Stores the left-most item from a \meta{comma list} in the
  \meta{token list variable} without removing it from the
  \meta{comma list}. The \meta{token list variable} is assigned locally.
  If the \meta{comma list} is empty the \meta{token list variable}
  is set to the marker value \cs{q_no_value}.
\end{function}

\begin{function}[TF, added = 2012-05-14]{\clist_get:NN, \clist_get:cN}
  \begin{syntax}
    \cs{clist_get:NNTF} \meta{comma list} \meta{token list variable} \Arg{true code} \Arg{false code}
  \end{syntax}
  If the \meta{comma list} is empty, leaves the \meta{false code} in the
  input stream.  The value of the \meta{token list variable} is
  not defined in this case and should not be relied upon.  If the
  \meta{comma list} is non-empty, stores the top item from the
  \meta{comma list} in the \meta{token list variable} without removing it
  from the \meta{comma list}. The \meta{token list variable} is assigned
  locally.
\end{function}

\begin{function}[updated = 2011-09-06]{\clist_pop:NN, \clist_pop:cN}
  \begin{syntax}
    \cs{clist_pop:NN} \meta{comma list} \meta{token list variable}
  \end{syntax}
  Pops the left-most item from a \meta{comma list} into the
  \meta{token list variable}, \emph{i.e.}~removes the item from the
  comma list and stores it in the \meta{token list variable}.
  Both of the variables are assigned locally.
\end{function}

\begin{function}{\clist_gpop:NN, \clist_gpop:cN}
  \begin{syntax}
    \cs{clist_gpop:NN} \meta{comma list} \meta{token list variable}
  \end{syntax}
  Pops the left-most item from a \meta{comma list} into the
  \meta{token list variable}, \emph{i.e.}~removes the item from the
  comma list and stores it in the \meta{token list variable}.
  The \meta{comma list} is modified globally, while the assignment of
  the \meta{token list variable} is local.
\end{function}

\begin{function}[TF, added = 2012-05-14]{\clist_pop:NN, \clist_pop:cN}
  \begin{syntax}
    \cs{clist_pop:NNTF} \meta{comma list} \meta{token list variable} \Arg{true code} \Arg{false code}
  \end{syntax}
  If the \meta{comma list} is empty, leaves the \meta{false code} in the
  input stream.  The value of the \meta{token list variable} is
  not defined in this case and should not be relied upon.  If the
  \meta{comma list} is non-empty, pops the top item from the
  \meta{comma list} in the \meta{token list variable}, \emph{i.e.}~removes
  the item from the \meta{comma list}. Both the \meta{comma list} and the
  \meta{token list variable} are assigned locally.
\end{function}

\begin{function}[TF, added = 2012-05-14]{\clist_gpop:NN, \clist_gpop:cN}
  \begin{syntax}
    \cs{clist_gpop:NNTF} \meta{comma list} \meta{token list variable} \Arg{true code} \Arg{false code}
  \end{syntax}
  If the \meta{comma list} is empty, leaves the \meta{false code} in the
  input stream.  The value of the \meta{token list variable} is
  not defined in this case and should not be relied upon.  If the
  \meta{comma list} is non-empty, pops the top item from the
  \meta{comma list} in the \meta{token list variable}, \emph{i.e.}~removes
  the item from the \meta{comma list}. The \meta{comma list} is modified
  globally, while the \meta{token list variable} is assigned locally.
\end{function}

\begin{function}
  {
    \clist_push:Nn,  \clist_push:NV,  \clist_push:No,  \clist_push:Nx,
    \clist_push:cn,  \clist_push:cV,  \clist_push:co,  \clist_push:cx,
    \clist_gpush:Nn, \clist_gpush:NV, \clist_gpush:No, \clist_gpush:Nx,
    \clist_gpush:cn, \clist_gpush:cV, \clist_gpush:co, \clist_gpush:cx
  }
  \begin{syntax}
    \cs{clist_push:Nn} \meta{comma list} \Arg{items}
  \end{syntax}
  Adds the \Arg{items} to the top of the \meta{comma list}.
  Spaces are removed from both sides of each item as for any
  \texttt{n}-type comma list.
\end{function}

\section{Using a single item}

\begin{function}[added = 2014-07-17, EXP]
  {\clist_item:Nn, \clist_item:cn, \clist_item:nn}
  \begin{syntax}
    \cs{clist_item:Nn} \meta{comma list} \Arg{integer expression}
  \end{syntax}
  Indexing items in the \meta{comma list} from~$1$ at the top (left), this
  function evaluates the \meta{integer expression} and leaves the
  appropriate item from the comma list in the input stream. If the
  \meta{integer expression} is negative, indexing occurs from the
  bottom (right) of the comma list. When the \meta{integer expression}
  is larger than the number of items in the \meta{comma list} (as
  calculated by \cs{clist_count:N}) then the function expands to
  nothing.
  \begin{texnote}
    The result is returned within the \tn{unexpanded}
    primitive (\cs{exp_not:n}), which means that the \meta{item}
    does not expand further when appearing in an \texttt{x}-type
    argument expansion.
  \end{texnote}
\end{function}

\section{Viewing comma lists}

\begin{function}[updated = 2015-08-03]{\clist_show:N, \clist_show:c}
  \begin{syntax}
    \cs{clist_show:N} \meta{comma list}
  \end{syntax}
  Displays the entries in the \meta{comma list} in the terminal.
\end{function}

\begin{function}[updated = 2013-08-03]{\clist_show:n}
  \begin{syntax}
    \cs{clist_show:n} \Arg{tokens}
  \end{syntax}
  Displays the entries in the comma list in the terminal.
\end{function}

\begin{function}[added = 2014-08-22, updated = 2015-08-03]{\clist_log:N, \clist_log:c}
  \begin{syntax}
    \cs{clist_log:N} \meta{comma list}
  \end{syntax}
  Writes the entries in the \meta{comma list} in the log file.  See
  also \cs{clist_show:N} which displays the result in the terminal.
\end{function}

\begin{function}[added = 2014-08-22]{\clist_log:n}
  \begin{syntax}
    \cs{clist_log:n} \Arg{tokens}
  \end{syntax}
  Writes the entries in the comma list in the log file.  See also
  \cs{clist_show:n} which displays the result in the terminal.
\end{function}

\section{Constant and scratch comma lists}

\begin{variable}[added = 2012-07-02]{\c_empty_clist}
  Constant that is always empty.
\end{variable}

\begin{variable}[added = 2011-09-06]{\l_tmpa_clist, \l_tmpb_clist}
  Scratch comma lists for local assignment. These are never used by
  the kernel code, and so are safe for use with any \LaTeX3-defined
  function. However, they may be overwritten by other non-kernel
  code and so should only be used for short-term storage.
\end{variable}

\begin{variable}[added = 2011-09-06]{\g_tmpa_clist, \g_tmpb_clist}
  Scratch comma lists for global assignment. These are never used by
  the kernel code, and so are safe for use with any \LaTeX3-defined
  function. However, they may be overwritten by other non-kernel
  code and so should only be used for short-term storage.
\end{variable}

\end{documentation}

\PrintIndex

\end{document}
