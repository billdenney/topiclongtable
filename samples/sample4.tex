\documentclass[11pt]{article}

% The package has no options.
\usepackage{topiclongtable}

\usepackage[
  paperwidth=5in,
  paperheight=5in,
  margin=.5in,
  %% showframe
]{geometry}

\begin{document}

% Our main environment is topiclongtable - which is
% longtable with additional functionality.
%
% Column specifications must prepend F to the first column
% and T to all other. Furthermore, the column numer is required
% as a second parameter.
\begin{topiclongtable}{|Fl|Tl|Tl|Tl|Tp{0.1in}|}{5}
  % To draw lines at the top and bottom of each table segment
  % we simply use empty head/foot. (This way we have maximum flexibility).
  \hline\endhead
  \hline\endfoot
  % Each row starts with \TopicLine (which draws the line) and has
  % a few columns of subtopics. Hierarchy is determined by either
  % omitting the topic label or repeating it.
  \TopicLine \Topic[T1] & \Topic[ST1] & \Topic[SST1] & X & Y \\
  \TopicLine \Topic     & \Topic      & \Topic[SST2] & X & Y Z A B C D E F G H bill foo bar baz\\ 
  \TopicLine \Topic     & \Topic[ST2] & \Topic[SST3] & X & Y Z A B C D E F G H bill foo bar baz\\ 
  \TopicLine \Topic[T2] & \Topic[ST3] & \Topic[SST4] & X & Y Z A B C D E F G H bill foo bar baz\\ 
  \TopicLine \Topic     & \Topic[ST4] & \Topic[SST5] & X & Y Z A B C D E F G H bill foo bar baz\\ 
  \TopicLine \Topic     & \Topic      & \Topic       & X & Y Z A B C D E F G H bill foo bar baz\\
  \TopicLine \Topic     & \Topic      & \Topic[SST5] & X & Y Z A B C D E F G H bill foo bar baz\\
  \TopicLine \Topic     & \Topic      & \Topic       & X & Y Z A B C D E F G H bill foo bar baz\\
  \TopicLine \Topic     & \Topic      & \Topic       & X & Y Z A B C D E F G H bill foo bar baz\\
  \TopicLine \Topic     & \Topic      & \Topic       & X & Y Z A B C D E F G H bill foo bar baz\\
  \TopicLine \Topic     & \Topic      & \Topic       & X & Y Z A B C D E F G H bill foo bar baz\\
  \TopicLine \Topic     & \Topic[ST5] & \Topic[SST6] & X & Y Z A B C D E F G H bill foo bar baz\\
  \TopicLine \Topic     & \Topic      & \Topic       & X & Y Z A B C D E F G H bill foo bar baz\\
  \TopicLine \Topic     & \Topic      & \Topic       & X & Y Z A B C D E F G H bill foo bar baz\\
  \TopicLine \Topic[T1] & \Topic[ST1] & \Topic[SST1] & X & Y Z A B C D E F G H bill foo bar baz\\
  \TopicLine \Topic     & \Topic      & \Topic[SST2] & X & Y Z A B C D E F G H bill foo bar baz\\ 
  \TopicLine \Topic     & \Topic      & \Topic[SST2] & X & Y Z A B C D E F G H bill foo bar baz\\ 
  \TopicLine \Topic     & \Topic      & \Topic[SST2] & X & Y Z A B C D E F G H bill foo bar baz\\ 
  \TopicLine \Topic     & \Topic      & \Topic[SST2] & X & Y Z A B C D E F G H bill foo bar baz\\ 
  \TopicLine \Topic     & \Topic      & \Topic[SST2] & X & Y Z A B C D E F G H bill foo bar baz\\ 
  \TopicLine \Topic     & \Topic      & \Topic[SST2] & X & Y Z A B C D E F G H bill foo bar baz\\ 
  \TopicLine \Topic     & \Topic      & \Topic[SST2] & X & Y \\ 
  \TopicLine \Topic     & \Topic      & \Topic[SST2] & X & Y \\
\end{topiclongtable}

% We can configure a continuation mark (empty by default) if we need it.
\TopicSetContinuationCode{\ (cont.)}

\begin{topiclongtable}{|Fl|Tl|Tr|}{3}
  \hline\endhead
  \hline\endfoot
  \TopicLine \Topic[R1] & \Topic[SR1] & XXX \\
  \TopicLine \Topic     & \Topic      & XX \\ 
  \TopicLine \Topic     & \Topic      & X \\ 
  \TopicLine \Topic     & \Topic      & X \\ 
  \TopicLine \Topic     & \Topic      & X \\ 
  \TopicLine \Topic     & \Topic      & XX \\ 
  \TopicLine \Topic     & \Topic[SR2] & X \\ 
  \TopicLine \Topic     & \Topic[SR2] & X \\ 
  \TopicLine \Topic     & \Topic      & XXXX \\ 
  \TopicLine \Topic     & \Topic      & X \\ 
  \TopicLine \Topic     & \Topic[SR2] & X \\ 
  \TopicLine \Topic     & \Topic[SR3] & XX \\ 
  \TopicLine \Topic     & \Topic      & X \\ 
  \TopicLine \Topic     & \Topic      & X \\ 
\end{topiclongtable}

\end{document}

